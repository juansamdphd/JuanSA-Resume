%%%%%%%%%%%%%%%%%
% This is an example CV created using altacv.cls (v1.1.5, 1 December 2018) written by
% LianTze Lim (liantze@gmail.com), based on the
% Cv created by BusinessInsider at http://www.businessinsider.my/a-sample-resume-for-marissa-mayer-2016-7/?r=US&IR=T
%
%% It may be distributed and/or modified under the
%% conditions of the LaTeX Project Public License, either version 1.3
%% of this license or (at your option) any later version.
%% The latest version of this license is in
%%    http://www.latex-project.org/lppl.txt
%% and version 1.3 or later is part of all distributions of LaTeX
%% version 2003/12/01 or later.
%%%%%%%%%%%%%%%%

%% If you are using \orcid or academicons
%% icons, make sure you have the academicons
%% option here, and compile with XeLaTeX
%% or LuaLaTeX.
% \documentclass[10pt,a4paper,academicons]{altacv}

%% Use the "normalphoto" option if you want a normal photo instead of cropped to a circle
% \documentclass[10pt,a4paper,normalphoto]{altacv}

\documentclass[10pt,a4paper,ragged2e]{altacv}

%% AltaCV uses the fontawesome and academicon fonts
%% and packages.
%% See texdoc.net/pkg/fontawecome and http://texdoc.net/pkg/academicons for full list of symbols. You MUST compile with XeLaTeX or LuaLaTeX if you want to use academicons.

% Change the page layout if you need to
\geometry{left=1cm,right=9cm,marginparwidth=6.8cm,marginparsep=1.2cm,top=1.25cm,bottom=1.25cm}

% Change the font if you want to, depending on whether
% you're using pdflatex or xelatex/lualatex
\ifxetexorluatex
  % If using xelatex or lualatex:
  \setmainfont{Carlito}
\else
  % If using pdflatex:
  \usepackage[utf8]{inputenc}
  \usepackage[T1]{fontenc}
  \usepackage[default]{lato}
\fi

% Change the colours if you want to
\definecolor{VividPurple}{HTML}{3E0097}
\definecolor{SlateGrey}{HTML}{2E2E2E}
\definecolor{LightGrey}{HTML}{666666}
\colorlet{heading}{SlateGrey}
\colorlet{accent}{LightGrey}
\colorlet{emphasis}{SlateGrey}
\colorlet{body}{LightGrey}

% Change the bullets for itemize and rating marker
% for \cvskill if you want to
\renewcommand{\itemmarker}{{\small\textbullet}}
\renewcommand{\ratingmarker}{\faCircle}

%% sample.bib contains your publications
\addbibresource{references.bib}

\begin{document}
\name{Juan C. Sanchez-Arias}
\tagline{MD (Universidad del Valle, '14) | PhD in Neuroscience (University of Victoria, '20)}
% Cropped to square from https://en.wikipedia.org/wiki/Marissa_Mayer#/media/File:Marissa_Mayer_May_2014_(cropped).jpg, CC-BY 2.0
\photo{2.5cm}{linkedin_pic}
\personalinfo{%
  % Not all of these are required!
  % You can add your own with \printinfo{symbol}{detail}
  \email{juan@juansanar.com}
%   \phone{000-00-0000}
  \mailaddress{545-4678 Elk Lake}
  \location{Victoria, BC, Canada}
  \homepage{www.juansanar.com/}
  \twitter{@juan\_sanar}
  \linkedin{linkedin.com/in/juancsanchezarias}
  \github{github.com/juansamdphd} % I'm just making this up though.
%   \orcid{orcid.org/0000-0000-0000-0000} % Obviously making this up too. If you want to use this field (and also other academicons symbols), add "academicons" option to \documentclass{altacv}
}

%% Make the header extend all the way to the right, if you want.
\begin{fullwidth}
\makecvheader
\end{fullwidth}

%% Depending on your tastes, you may want to make fonts of itemize environments slightly smaller
\AtBeginEnvironment{itemize}{\small}

%% Provide the file name containing the sidebar contents as an optional parameter to \cvsection.
%% You can always just use \marginpar{...} if you do
%% not need to align the top of the contents to any
%% \cvsection title in the "main" bar.
\cvsection[page1sidebar]{Experience}

\cvevent{Graduate Research Fellow}{University of Victoria}{Jan. 2015 -- Apr. 2020}{Victoria, BC. Canada}
\begin{itemize}
  \item {Advisor: Leigh Anne Swayne, PhD.}
  \item {Area of Study: Area of study: Pannexin 1 channels, dendritic spine plasticity, synapse formation, channel trafficking, neuronal cytoskeleton dynamics, neural stem cells, advanced microscopy for cell biology.}
  \begin{itemize}
      \item {Generated conditional and conditional-inducible knockout models for the study of cerebral cortex development.}
      \item {Optimized protocols to generate primary neuronal cultures from neonatal mice suitable for network analysis.}
      \item {Developed methods to visualize dendritic spines and filopodia in tissue sections and living primary neurons.}
      \item {Established immunocyto(histo)chemistry protocols that preserve the neuronal cytoskeleton.}
      \end{itemize}
\end{itemize}

\cvevent{Pre-Diploma Rotatory Medicine \& Surgery Internship}{School of Medicine - Uniersidad del Valle}{Sept. 2013 -- Sept 2014}{Cali, Valle. Colombia}

  \begin{itemize}
  \item Hospital Universitario del Valle ESE - Universidad del Valle.
   \end{itemize}

\cvevent{Research Intern in Biomedical Sciences - Neuroscience}{Centro de Estudios Cerebrales - Universidad del Valle}{Feb. 2014 -- July 2014}{Cali,Valle.  Colombia}
\begin{itemize}
  \item {Advisors: Prof. Martha Escobar, MSc; Prof. Hernan Pimienta, MSc, Prof. Efrain Buritica, MSc, PhD.}
  \item {Area of study: functional neuroanatomy, cerebral cortex organization, traumatic brain injury, stroke, neuroprotection.}
  % \item {Performed carotid ligation in Whistar rats using microsurgery techniques.}
  % \item {Used the weight-drop model of diffuse traumatic brain injury to generate organotypic slice cultures from rats.}
  % \item {Optimized immunohystochemistry for neuronal and astrocytic markers in thick rat brain tissue sections.}
\end{itemize}

\cvevent{Student Researcher}{School of Public Health - Universidad del Valle}{Aug. 2010 -- Feb. 2011}{Cali, Valle. Colombia}
\begin{itemize}
  \item {Advisors: Enrique A. Esteves-Rivera, MD and Elsa P. Muñoz, MD, MPH}
  \item {Area of study: Cardiovascular risk factor assessment in spinal cord injured patient assisting to a tertiary-level hospital.}
  \begin{itemize}
    \item {Prepared research project proposals, liaised with ethical boards, and established a network of collaborators to complete the study using a standarized survey.}
    \item {Prepared data management plan, data collection, and statistical analysis.}
    \item {Contributed to the assessment and management of patients with chronic spinal cord injury.}
  \end{itemize}
\end{itemize}

% \divider

% \cvevent{Product Engineer}{Google}{23 June 1999 -- 2001}{Palo Alto, CA}

% \begin{itemize}
% \item Joined the company as employe \#20 and female employee \#1
% \item Developed targeted advertisement in order to use user's search queries and show them related ads
% \end{itemize}

\cvsection{A Day of My Life}

% Adapted from @Jake's answer from http://tex.stackexchange.com/a/82729/226
% \wheelchart{outer radius}{inner radius}{
% comma-separated list of value/text width/color/detail}
% Some ad-hoc tweaking to adjust the labels so that they don't overlap
\wheelchart{1.5cm}{0.5cm}{%
  10/10em/accent!30/Sleeping \& dreaming about work,
  25/9em/accent!60/Planing-doing-writing-reading science,
  5/13em/accent!10/\footnotesize\\Practicing programming languages,
  20/15em/accent!40/Time for family-friends and mentoring,
  5/8em/accent!20/\footnotesize Learning data science and sharpening statistical analysis methods,
  30/9em/accent/Managing multiple research projects,
  5/8em/accent!20/Working out
}

\clearpage

\cvsection[page2sidebar]{Publications}

\nocite{swayne2016}
\nocite{wicki-stordeur2016}
\nocite{sanchez-arias2016}
\nocite{gunton2017}
\nocite{xu2018}
\nocite{epp2019}
\nocite{sanchez-arias2019}
\nocite{choi2019}
\nocite{frederiksen2019}
\nocite{chen2019}
\nocite{sanchez-arias2020}
% \printbibliography[heading=pubtype,title={\printinfo{\faBook}{Books}},type=book]

% \divider

\printbibliography[heading=none,title={\printinfo{\faFileTextO}{Journal Articles}}, type=article]

\divider

\printbibliography[heading=pubtype,title={\printinfo{\faGroup}{Conference Proceedings}},type=inproceedings]

%% If the NEXT page doesn't start with a \cvsection but you'd
%% still like to add a sidebar, then use this command on THIS
%% page to add it. The optional argument lets you pull up the
%% sidebar a bit so that it looks aligned with the top of the
%% main column.
% \addnextpagesidebar[-1ex]{page3sidebar}


\end{document}
